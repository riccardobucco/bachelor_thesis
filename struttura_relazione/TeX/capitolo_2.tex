\part{Presentazione delle motivazioni, delle finalità e dei vincoli dello stage}
	\section{Gli stage nella strategia aziendale}
		\subsection{Motivazioni e obiettivi aziendali}
			Descrivo quali sono le motivazioni e gli obiettivi che portano l'azienda ad assumere uno stagista. In seguito passo al mio
			caso particolare e descrivo perchè l'azienda abbia deciso di intraprendere lo stage che ho svolto.
		\subsection{Integrazione e formazione degli stagisti}
			Descrivo i modi in cui in genere l'azienda forma gli stagisti per fare in modo che imparino e si adattino in fretta alla
			realtà aziendale. Descrivo inoltre come l'azienda si preoccupa di integrare gli stagisti nel resto del team.
			Passo poi al mio caso particolare e descrivo come questo sia stato attuato durante la mia esperienza.
		\subsection{Gestione degli obiettivi di stage}
			Descrivo come l'azienda individui gli obiettivi minimi che lo stagista deve raggiungere durante la sua esperienza. Aggiungo
			inoltre come tali obiettivi vengano mano a mano adattati nel caso in cui il lavoro dello stagista proceda più lentamente o
			velocemente del previsto. Descrivo i modi in cui l'azienda interviene nel caso in cui le cose non stiano andando come
			previsto. Infine parlo del mio caso particolare e parlo di come i miei obiettivi siano stati gestiti durante lo stage.
	\section{Descrizione del progetto di stage}
		\subsection{Prodotti}
			\subsubsection{PastBook Mobile SDK}
				Descrivo l'SDK che devo realizzare utilizzando le API e i services messi a disposizione da pastbook. Descrivo le
				funzionalità minime (e massime) che tale prodotto deve avere. Inquadro tale prodotto all'interno degli obiettivi
				aziendali futuri.
			\subsubsection{My Year Photo Book}
				Descrivo il prodotto principale che deve essere realizzato durante lo stage, un'applicazione per iphone e android
				che replichi le funzionalità offerte dall'azienda nel sito web. Descrivo come tale prodotto faccia uso dell'SDK
				di cui ho parlato in precedenza. Descrivo come tale prodotto sia fondamentale per la strategia aziendale e in
				generale per perseguire gli obietivi di cui ho parlato in precedenza.
		\subsection{Tecnologie}
			\subsubsection{Appcelerator Platform}
				Descrivo tutti i prodotti appartenenti alla piattaforma Appcelerator che mi è stato richiesto di utilizzare
				(Titanium SDK e Appcelerator CLI) e che ho scelto personalmente di utilizzare (Titanium Studio e Alloy Framework).
				Per ognuno di essi indico lo scopo e l'ambito in cui sono utilizzati. Descrivo infine i pregi di tali tecnologie
				rispetto ad altre simili.
			\subsubsection{Stripe}
				Descrivo brevemente la suite di API che l'azienda ha deciso di utilizzare per implementare un form di checkout
				all'interno dell' applicazioni mobile. Descrivo i vantaggi che tale tecnologia ha rispetto ad altre simili.
			\subsubsection{Hockeyapp}
				Descrivo la piattaforma che l'azienda ha deciso di utilizzare per la distribuzione dell'applicazione durante il
				beta testing e per il report dei crash dell'applicazione stessa. Descrivo l'utilità che essa ha durante l'analisi
				dei dati e del bug fixing.
			\subsubsection{Google Analytics e Google Tag Manager}
				Descrivo la tecnologia utilizzata dall'azienda all'interno di altri prodotti (google analytics) e quella che mi è
				stato richiesto di utilizzare (google tag manager) all'interno dell'applicazione per poter inviare dati utili
				riguardanti l'uso che gli utenti fanno di essa. Descrivo come i dati inviati tramite il tag manager possano essere
				elaborati, utilizzati e visualizzati utilizzando google analytics.
		\subsection{Scadenze}
			Descrivo i vincoli temporali che l'azienda si aspetta che io segua. Faccio riferimento al fatto che tali vincoli sono molto
			importanti per l'azienda in quanto essa si aspetta di utilizzare i prodotti da me realizzati fin da subito, durante il
			periodo che precede le feste natalizie.
		\subsubsection{Metodologia di lavoro}
			Descrivo i vincoli metodologici, ovvero la metodologia di lavoro che sono tenuto a seguire in azienda per poter collaborare
			al meglio con gli altri membri del team. Descrivo come ci si aspetta che io lavori per poter controllare che gli obiettivi
			vengano raggiunti e le scadenze (di cui ho parlato in precedenza) siano rispettate.
	\section{Scelta dello stage}
		\subsection{Obiettivi personali}
			\subsubsection{Sviluppo della capacità di collaborazione}
				Racconto come il mio primo obiettivo durante lo svolgimento di un generico stage fosse quello di migliorare nel modo
				maggiore possible la capacità di collaborare con i colleghi (cosa scarsamente sviluppata durante gli studi ma
				fondamentale nel mondo del lavoro).
			\subsubsection{Potenziamento dell'autonomia}
				Descrivo come un secondo importante obiettivo fosse raggiungere una maggiore maturità nel prendere personalmente
				decisioni durante lo svolgimento dei vari compiti a me assegnati, senza dover chiedere continuamente aiuto ad altri.
				Spiego come questo obiettivo non contraddica il precedente e anzi possano essere perseguiti entrambi in parallelo.
			\subsubsection{Accrescimento dello spirito critico}
				Descrivo l'ultimo ma non meno importante obiettivo: essere in grado di sviluppare uno spirito critico (insegnatomi
				all'università) nei confronti di qualsiasi cosa io impari, per poter valutare se e quanto questa sia buona. Aggiungo
				come tale capacità sia fondamentale in quanto oggi le tecnologie e gli strumenti a nostra disposizione aumentano a
				dismisura.
		\subsection{Motivazioni per cui ho intrapreso lo stage}
			\subsubsection{Ambiente internazionale}
				Descrivo come sia stato decisivo il fatto che l'ambiente fosse internazionale e composto da persone con competenze,
				obiettivi, passioni e anche culture completamente differenti. Descrivo come mi abbia attirato poter intraprendere
				un'esperienza più vasta e che non comprendesse solo l'inserimento nel mondo del lavoro ma anche l'entrare a contatto
				con idee, pensieri e opinioni di un immenso ecosistema di startup quale è Amsterdam. Descrivo come la presenza di
				persone con background molto differenti si sposasse perfettamente con il mio obiettivo di migliorare le capacità di
				collaborare che mi ero posto.
			\subsubsection{Utilizzo di Titanium}
				Descrivo come mi abbia attratto molto il framework Titanium. Aggiungo come in particolare mi abbia attratto il
				progetto di poter sviluppare in un solo linguaggio (peraltro già largamente usato in molti ambiti) la stessa
				applicazione per sistemi operativi diversi. Aggiungo come, nonostante fossi a conoscenza dei problemi che un tale
				framework 
			\subsubsection{Dimensione ridotta dell'azienda e informalità dei rapporti}
				Descrivo come mi abbia spaventato essere subito inserito all'interno di un'azienda medio/grande per paura di non
				essere sufficientemente seguito nei momenti di difficoltà. Temevo inoltre di non poter avere una visione d'insieme
				dell'ambiente in cui ero inserito e quindi di essere in qualche modo isolato al solo "reparto sviluppo" senza avere
				la possibilità di imparare come competenze e compiti diversi coesistono e collaborano. Descrivo come il fatto che
				l'azienda fosse di dimensioni ridotte e i rapporti tra colleghi informali mi abbiano convinto da questo punto di
				vista. Aggiungo infine come il fatto di essere in pochi richiedesse che io fossi fortemente autonomo nel mio lavoro,
				cosa che combaciava con uno dei miei obiettivi.
		\subsection{Aspettative}
			\subsubsection{Rielaborazione, adattamento e approfondimento dei concetti studiati}
				Descrivo come, prima di cominciare lo stage, fossi convinto che avrei utilizzato fortemente i concetti e le nozioni
				che ho imparato e studiato durante gli anni universitari, adattandoli ai problemi che avevo di fronte e
				approfondendoli personalmente o con l'aiuto del tutor.
			\subsubsection{Instaurazione di contatti per future collaborazioni}
				Descrivo come mi aspettassi di entrare in contatto con molte persone e aziende (essendo la mia azienda inserita in
				un contesto molto dinamico di startup e realtà consolidate). A partire da questi rapporti mi aspettavo di poter
				costruire collaborazioni ed eventualmente possibilità di lavoro concreto.
