\part{Resoconto retrospettivo}
	\section{Resoconto del raggiungimento degli obiettivi aziendali}
		\subsection{Funzionalità del prodotto}
			Elenco quali obiettivi posti inizialmente dall'azienda sono stati raggiunti e quali invece no. In questa sezione parlo delle
			funzionalità che l'azienda aveva deciso di inserire all'interno del prodotto che mi è stato assegnato.
		\subsection{Formazione e integrazione}
			Descrivo quali obiettivi riguardanti la mia formazione siano stati raggiunti. Tale formazione riguarda sia le tecnologie e
			gli strumenti che l'azienda mi ha messo a disposizione che il modo di lavorare che si è cercato di trasmettermi. Descrivo
			l'esito del tentativo dell'azienda di fare in modo che mi integrassi con il resto del team, portando nuove idee anche al di
			fuori dell'ambito in cui lavoravo. Descrivo il perchè posso affermare che certi obiettivi siano stati raggiunti
	\section{Bilancio dell'esperienza personale e professionale}
		\subsection{Obiettivi raggiunti}
			Indico quali obiettivi personali sono riuscito a raggiungere tra quelli che mi ero posto inizialmente, prima di iniziare
			lo stage. Descrivo in cosa sono riuscito a migliorare e perchè posso dire di essere migliorato.
		\subsection{Aspettative soddisfatte}
			Indico se le aspettative che avevo prima di cominciare sono state soddisfatte oppure no. Descrivo ciò che è stato esattamente
			come mi aspettavo e cosa invece mi ha sorpreso o in generale è stato inaspettato.
		\subsection{Valore dello stage}
			\subsubsection{Ponte tra università e mondo del lavoro}
				Descrivo come lo stage abbia avuto un'importanza molto maggiore rispetto a quello che ero convinto avesse per
				immettermi nel mondo del lavoro. Descrivo come mi sia reso conto che, nonostante i numerosi strumenti che
				l'università mi offre, io abbia ancora molto da imparare (sia per quanto riguarda le conoscenze che per quanto
				riguarda la metodologia di lavoro). Indico infine come lo stage mi abbia permesso di capire che gli obiettivi nel
				mondo del lavoro sono molto differenti rispetto a quelli dell'ambito accademico: molto spesso sono dovuto giungere
				a compromessi e rinunciare a principi che ritenevo intoccabili in precedenza.
			\subsubsection{Prospettive ed idee differenti}
				Descrivo come l'ambiente (l'ecosistema di aziende nelle quali PastBook era inserita) e le circostanze nelle quali la
				mia esperienza si è svolta mi abbiano permesso di incontrare molte persone con le quali condividere idee, chiedere
				aiuto o semplicemente discutere su alcune nostre opinioni. Aggiungo come questa cosa mi abbia permesso di ottenere
				numerosi spunti sia per il prodotto che stavo realizzando che per aiutarmi a vedere molte cose sotto punti di vista
				differenti (non solo il punto di vista strettamente informatico, ma anche economico, sociale, comunicativo).
	\section{Considerazioni personali sul corso di studi e il mondo del lavoro}
		Espongo le mie considerazioni riguardanti quanto il corso di studi che ho frequentato sia appropriato in vista dell'inserimento
		nel mondo del lavoro. Descrivo i suoi pregi e i suoi difetti, includendo dove necessario dei consigli che mi sento di dare per
		poter migliorare l'attuale situazione.
