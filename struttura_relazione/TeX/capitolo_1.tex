\chapter{Analisi del contesto aziendale}
	\section{L'azienda e il suo ambito di attività}
		In tale sezione riporto innanzitutto il nome dell'azienda, dove essa si trova (sede legale e quella operativa) e quando è nata. In
		seguito descrivo brevemente di cosa si occupa l'azienda.
	\section{Prodotti offerti}
		\subsection{Photo Books in un click}
			In questa sezione descrivo il prodotto principale dell'azienda: un libro fotografico creato nel minor tempo possibile a
			partire da una delle tante sorgenti a disposizione. Il punto chiave sta nel fatto che esso è creato in modo automatico ed è
			quindi immediatamente disponibile per essere visualizzato, stampato e comprato.
		\subsection{Photo Books personalizzati}
			In questa sezione descrivo il prodotto dedicato agli utenti con esigenze maggiori: un libro fotografico ideato da zero,
			personalizzato e preparato a mano da un grafico professionista.
	\section{Organizzazione aziendale}
		\subsection{Obiettivi e risultati attesi}
			\subsubsection{Qualità del prodotto}
				In questa sezione descrivo il fatto che l'azienda intenda rendere il prodotto offerto quanto più possibile conforme
				alle aspettative del cliente. L'azienda, allo stesso tempo, intende rendere disponibile il prodotto a un prezzo
				inferiore rispetto alla concorrenza.
			\subsubsection{Semplicità di utilizzo degli strumenti}
				L'azienda desidera fortemente che gli strumenti messi a disposizione dei clienti per la creazione del prodotto siano
				più semplici possibile. E' importantissimo il fatto che il cliente ottenga esattamente ciò che vuole senza il bisogno
				di dover intervenire e modificare a mano il libro creato
			\subsubsection{Valorizzazione del capitale umano}
				L'azienda intende fare in modo che i dipendenti siano costantemente motivati ma soprattutto coinvolti. Si punta
				molto sulle relazioni umane tra i vari membri del team. In secondo l'azienda desidera che ciascun membro del team
				diventi il più possibile autonomo nelle sue mansioni.
		\subsection{Risorse disponibili}
			Descrivo le risorse umane, gli spazi e i materiali a disposizione dell'azienda.
		\subsection{Componenti e relazioni organizzative}
			\subsubsection{Funzioni e ruoli}
				Qui descrivo quali sono i ruoli che sono svolti all'interno dell'azienda (un dipendente può ricoprire più ruoli)
				affinchè essa possa funzionare correttamente
			\subsubsection{Struttura organizzativa}
				Descrivo come è strutturata l'azienda, come non vi sia sostanzialmente una gerarchia, come essa sia suddivisa in
				aree di lavoro con obiettivi comuni. Descrivo inoltre come spesso si creino dei gruppi di lavoro temporanei e non
				descrivibili nell'organigramma per portare a termine determinati compiti o per risolvere alcuni problemi. Intendo
				inoltre descrivere come l'azienda sia fortemente dinamica e non soggetta a regole specifiche nell'interazione tra
				aree di lavoro differenti.
			\subsubsection{Coordinamento e controllo}
				Descrivo come viene controllata e coordinata l'azienda. In particolare descrivo come l'azienda usi il metodo scrum,
				con standup giornalieri nei quali ognuno spiega cosa ha fatto, cosa intende fare, se ha particolari problemi o se è
				disponibile ad aiutare un collega. Descrivo inoltre le più corpose riunioni mensili nei quali ognuno è invitato a
				portare la propria idea in relazione ai vari problemi che l'azienda si ritrova ad affrontare. Descrivo inoltre quale
				tecnologia viene utilizzata per fare in modo si sappia sempre ognuno cosa sta facendo (sempre seguendo lo scrum).
			\subsubsection{Gestione del personale}
				\paragraph{Selezione, valutazione e formazione}
					Descrivo il fatto che per l'azienda risulti molto importante il fatto che una persona sia indipendente nel
					lavorare e soprattutto nel valutare un problema e nel prendere personalmente una decisione sensata.
					Descrivo inoltre come l'azienda attui percorsi di affiancamento e formativi per la persona che entra a far
					parte del gruppo (consentono di capire sia il modo di lavorare ma anche credenze, valori, regole che di
					solito non sono formalizzate).
				\paragraph{Motivazione e coinvolgimento}
					Descrivo come l'azienda cerchi in modo costante di fare in modo che ognuno si senta coinvolto nel progetto
					(soprattutto intraprendendo spesso confronti su problemi delle tipologie più disparate per fare in modo che
					ognuno possa presentare la propria idea e/o soluzione).
					Descrivo come si cerchi continuamente di fare in modo che si creino legami anche di amicizia fra colleghi,
					proponendo spesso attività terze da svolgere tutti insieme, eventualmente anche durante l'orario di lavoro.
			\subsubsection{Gestione dei clienti}
				Descrivo come vengono gestiti i clienti in azienda. In particolare mi soffermo sul fatto che la gestione dei vecchi
				e dei potenziali clienti avviene su due fronti: assistenza personalizzata per problemi durante la creazione dei libri
				e mantenimento di un blog nel quale vengono inseriti sia suggerimenti su come usare gli strumenti messi a
				disposizione dall'azienda sia fatti, eventi e curiosità che possono stuzzicare la mente del lettore. L'azienda
				prevede del personale dedicato esclusivamente a ciascuna di queste due aree.
			\subsubsection{Comunicazione nell'azienda}
				Qui descrivo sia il modo in cui si comunica in azienda (prevalentemente contatti informali, fortemente basati sulle
				relazioni, senza la presenza di una struttura gerarchica o di un forte grado di controllo) sia gli strumenti e/o le
				tecnologie utilizzati per fare ciò.
	\section{Sviluppo software}
		\subsection{Metodologia di lavoro}
			\subsubsection{Tipico ciclo di vita del software}
				Descrizione generale di come nell'azienda un'idea nasca, cresca e venga infine sviluppata. In particolare sottolineo
				come un'idea derivi solitamente da uno dei meeting mensili fatti tutti insieme. Proseguo dicendo come l'idea viene
				sviluppata e come si crei eventualmente un piccolo gruppo che prosegue nella definizione e nello sviluppo dei
				concetti elaborati dall'intero team. Dico infine che in base alla tipologia di prodotto questo venga testato solo
				internamente o anche con l'aiuto di beta tester esterni all'azienda.
			\subsubsection{Analisi dei requisiti}
				Descrivo come si svolge l'analisi dei requisiti. Essa è molto informale ed è aperta non solo agli sviluppatori, ma
				anche ad altri ruoli aziendali. Descrivo inoltre come l'informale "documentazione" (immagini, diagrammi, appunti)
				viene prodotta e poi memorizzata.
			\subsubsection{Progettazione e codifica}
				Descrivo la progettazione e la codifica, che in azienda arrivano quasi a coincidere (non esiste un momento in cui
				si pensi solo alla progettazione e di conseguenza non esiste nemmeno una documentazione di qualsiasi tipo che la
				riguardi, con tutti i problemi del caso). Descrivo come gli sviluppatori collaborino durante tali attività.
			\subsubsection{Beta testing}
				Alcuni prodotti e/o strumenti realizzati vengono distribuiti dall'azienda a delle persone incaricate di testarli
				e di rilasciare dei feedback, in modo tale che si possano apportare dei miglioramenti. Questa è l'unica forma di
				test che viene fatta in modo sistematico. Descrivo in questa sezione chi ha accesso ai prodotti in anteprima, quali
				sono gli strumenti che lo permettono e come si analizzano i dati provenienti dai feedback.
		\subsection{Tecnologie, tecniche e strumenti utilizzati}
			\subsubsection{Generazione di un Photo Book}
				Descrivo in modo molto generale come l'azienda produce il pdf che rappresenta il libro fotografico che poi viene
				successivamente stampato. Descrivo inoltre le tecnologie che sono utilizzate durante questo processo.
			\subsubsection{API e services}
				Comincio con il descrivere brevemente il server dell'azienda (tecnologie utilizzate e architettura generale). In
				seguito mi concentro su due componenti fondamentali di tale architettura che sono a disposizione degli sviluppatori:
				API REST (richieste basilari per creare/modificare un album fotografico) e services (richieste complesse e
				strutturate per l'esecuzione automatica di alcuni compiti).
			\subsubsection{Ambiente di lavoro virtuale}
				Descrivo come gli sviluppatori utilizzino un ambiente di lavoro virtuale che simuli in tutto e per tutto quello
				reale, in modo tale da non dover lavorare direttamente sui prodotti accessibili al pubblico. Descrivo le tecnologie
				utilizzate per la creazione di tale ambiente.
	\section{Propensione all'innovazione}
		Descrivo il rapporto che l'azienda ha con l'innovazione, in relazione anche alla tipologia di utenti alla quale si rivolge.
