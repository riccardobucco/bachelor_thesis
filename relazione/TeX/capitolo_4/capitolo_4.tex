\chapter{Resoconto retrospettivo}
	\section{Resoconto del raggiungimento degli obiettivi aziendali}
		\subsection{Funzionalità dei prodotti}
			Sono riuscito a realizzare tutte le funzionalità obbligatorie che l'azienda voleva fossero presenti all'interno dei prodotti
			PastBook Mobile SDK e My Year Photo Book. In particolare, l'SDK comprende componenti che gestiscano le seguenti funzioni:
			\begin{itemize}
				\item autenticazione e autorizzazione a Facebbok e Instagram;
				\item accesso alle API aziendali;
				\item accesso ai Service aziendali;
				\item controllo e correzione di eventuali errori.
			\end{itemize}
			L'applicazione, invece, permette agli utenti di:
			\begin{itemize}
				\item realizzare Photo Book a partire dalle immagini tratte da Facebook o Instagram;
				\item visualizzare il Photo Book creato;
				\item pagare il Photo Book creato.
			\end{itemize}
			Ho realizzato anche tutte le funzionalità ritenute opzionali dall'azienda: esse rappresentano il vero valore aggiunto dei
			prodotti sviluppati durante lo stage. In particolare, all'SDK è stato aggiunto il Launcher, che permette agli sviluppatori di
			non preoccuparsi del processo di creazione dei Photo Book. All'applicazione, invece, è stata aggiunta la possibilità di
			apportare modifiche agli album e di intrattenere gli utenti durante l'attesa.
		\subsection{Formazione e integrazione}
			Posso affermare che l'azienda abbia avuto successo nel fare in modo che io acquisissi le capacità e le abilità necessarie
			all'attività lavorativa che mi era stata assegnata. In particolare, il mio tutor ha cercato di rendermi autonomo fornendomi
			indicazioni e strumenti utili per assimilare velocemente le conoscenze necessarie. Inoltre, egli mi ha spesso lasciato
			sbagliare, intervenendo con azioni correttive piuttosto che preventive: ciò mi ha permesso di imparare dai miei errori.\\
			Sono riuscito a padroneggiare gli strumenti e le tecnologie necessarie per il progetto assegnatomi:
			\begin{itemize}
				\item Ho una profonda conoscenza dei vari aspetti legati all'autenticazione ai social network Facebook e Instagram.
				So utilizzare propriamente tutte le API che essi mettono a disposizione. Ho appreso anche alcuni elementi
				riguardanti l'uso degli SDK per Android e iOS.
				\item Ho imparato i concetti fondamentali riguardanti l'uso di metodi di pagamento (nel mio caso utilizzando Stripe).
				\item Ho capito quali sono i passaggi da svolgere per rilasciare un'applicazione. Ho imparato a padroneggiare gli
				strumenti utili a questo scopo.
				\item Ho approfondito molto l'utilizzo di Appcelerato Titanium e di molti prodotti ad esso collegati. Ho sviluppato
				un profondo spirito critico nei suoi confronti, apprezzandone i pregi ma riconoscendo anche i molti difetti.
				\item Ho imparato ad utilizzare strumenti che permettono di testare le applicazioni mobile.
			\end{itemize}
			Mi sono fortemente adattato alla metodologia di	lavoro utilizzata in azienda, al punto tale che verso la fine
			dell'esperienza di stage partecipavo solo raramente a momenti di confronto con il mio tutor e lavoravo al pari dei miei
			colleghi. In particolare:
			\begin{itemize}
				\item ho cominciato a capire quali esigenze e quali motivazioni hanno portato l'azieda ad adottare metodi agili;
				\item ho capito quali benefici e quali rischi comporta l'adozione di metodi agili.
			\end{itemize}
			L'azienda si è anche impegnata molto nel fare in modo che io mi ambientassi velocemente. In particolare, sono stato invitato
			fin da subito a partecipare in modo attivo ai meeting durante i quali si proponevano strategie da adottare nei vari ambiti
			in cui PastBook è impegnata. Sono spesso riuscito a sostenere una mia idea, proponendola e spiegandola ai miei colleghi. Ciò
			mi ha permesso di entrare a contatto con vari aspetti dell'azienda.
	\section{Bilancio dell'esperienza personale e professionale}
		\subsection{Obiettivi raggiunti}
			Al termine stage, posso affermare di aver raggiunto pienamente tutti gli obeittivi che mi ero posto prima di cominciare
			questa esperienza. Infatti, anche se ho dovuto affrontare alcune difficoltà, sono riuscito a sfruttare le varie situazioni
			a mio vantaggio, in modo tale da imparare il più possibile.\\
			\begin{center}
				\rowcolors{1}{azzurro}{azzurro_chiaro}
				\begin{tabular}[H]{p{0.25\textwidth} | p{0.60\textwidth}}
					Sviluppo capacità di collaborazione &
					Obiettivo raggiunto perché ho cooperato attivamente con colleghi in possesso di competenze completamente
					differenti dalle mie.\\
					\hline
					Potenziamento dell'autonomia &
					Obiettivo raggiunto grazie alla forte responsabilità assegnatami dall'azienda.\\
					\hline
					Accrescimento dello spirito critico &
					Obiettivo raggiunto in seguito alle numerose scelte che ho dovuto effettuare a livello tecnologico ed
					implementativo.\\
				\end{tabular}
			\end{center}
			Prima di cominciare lo stage, mi ero posto l'obiettivo di aumentare la mia scarsa capacità di collaborazione. A fine
			esperienza, posso dire che, da questo punto di vista, ho raggiunto risultati decisamente soddisfacenti. Infatti, ho
			avuto l'occasione di lavorare sia con colleghi della mia area lavorativa (sviluppatori e CTO) che con persone che avevano
			capacità completamente diverse dalle mie (designer e addetti al marketing). In qualsiasi caso, dovevo riuscire a raggiungere
			un compromesso tra la mia idea e quella del mio interlocutore, che spesso e volentieri era differente da quella che
			proponevo. In particolare, mi sono dovuo scontrare con esigenze completamente diverse dalle mie: ho dovuto imparare a dare
			loro importanza durante l'esecuzione delle mie attività.\\
			Un secondo obiettivo che mi ero posto era quello di potenziare la mia autonomia nel lavoro. Anche questo risultato è stato
			ampiamente raggiunto. Infatti, fin da subito mi è stata data la possibilità di effettuare scelte rilevanti durante lo
			sviluppo del prodotto. Inoltre, nonostante la mia scarsa esperienza lavorativa, ho svolto gran parte delle attività
			singolarmente, senza l'aiuto di nessuno. Ho infine potuto gestire il mio tempo come preferivo, a patto di rispettare certi
			vincoli o scadenze poste dall'azienda.\\
			L'ultimo mio scopo era quello di accrescere il mio spirito critico in qualsiasi cosa facessi. Posso dire di essere
			soddisfatto anche da questo punto do vista. Da una parte ho dovuto effettuare varie scelte implementative, valutando i pro e
			i contro di ogni possibile soluzione. Dall'altra ho dovuto affrontare numerosi problemi dovuti alle tecnologie utilizzate:
			ho fatto delle scelte che si sono rivelate essere pericolose, ma da questi errori ho imparato a porre più attenzione in quel
			che facevo.
		\subsection{Aspettative soddisfatte}
			Non tutte le aspettative che avevo prima di cominciare l'esperienza di stage si sono realizzate. Ciò deriva fondamentalmente
			dal fatto che non avevo esperienza lavorativa; non sapevo dunque come si sarebbe rivelato essere il mondo del lavoro.\\
			\begin{center}
				\rowcolors{1}{azzurro}{azzurro_chiaro}
				\begin{tabular}[H]{p{0.25\textwidth} | p{0.60\textwidth}}
					Rielaborazione, adattamento e approfondimento dei concetti studiati &
					Aspettativa soddisfatta solo in parte, a causa dell'approccio poco formale che ho dovuto utilizzare durante
					la mia permanenza in azienda.\\
					\hline
					Instaurazione di contatti per future collaborazioni &
					Aspettativa soddisfatta perché ho potuto incontrare e consocere molte persone con le quali scambiare idee e
					conoscenze.\\
				\end{tabular}
			\end{center}
			Innanzitutto, mi aspettavo che avrei fatto un grande uso delle nozioni e dei concetti appresi durante lo studio; credevo
			inoltre che un approccio analitico al problema sarebbe stato ben accetto da parte dell'azienda. Ciò è accaduto solo in parte.
			Infatti, posso affermare che l'azienda fosse completamente contraria all'approccio “scolastico” ai problemi: ciò
			che contava era il risultato, magari ottenuto in tempi molto brevi. Dunque, poco importava se a livello teorico un prodotto
			non fosse ben realizzato: l'imporante era che esso funzionasse e generasse profitti immediati. Questo modo di lavorare mi ha
			stupito parecchio e non mi è stato facile adattarmi. D'altra parte, però, posso affermare che la conoscenza acquisita
			durante gli studi mi è stata molto utile, come da aspettative. Infatti, nonostante i metodi imposti da PastBook, ho
			utilizzato le mie conoscenze in modo da prevenire molte delle situazioni pericolose in cui sarei potuto incorrere
			procedendo in modo poco ortodosso. In altre parole, ho utilizzato quanto sapevo per evitare errori piuttosto che per mettere
			in atto soluzioni ottimali, per le quali non avevo tempo.\\
			Una seconda aspettativa che avevo era di instaurare numerosi contatti per future collaborazioni. Da questo punto di vista
			sono stato contento. PastBook, infatti, è inserita in un contesto molto dinamico di startup e aziende consolidate, e questo
			mi ha dato l'opportunità di conoscere molte persone e aziende che cercassero personale al quale offrire concrete opportunità
			di lavoro.
		\subsection{Valore dello stage}
			\subsubsection{Ponte tra università e mondo del lavoro}
				L'esperienza di stage effettuata ha avuto un'importanza molto maggiore rispetto a quella che ero convinto avesse per
				immettermi nel mondo del lavoro. Mi sono accorto, infatti, che le tecniche e le metodologie di lavoro che molte
				aziende utilizzano sono completamente diverse da quelle proposte e insegnate negli ambienti universitari. Lo stage,
				da questo punto di vista, è stata una possibilità di sperimentare tali metodi.\\
				Lo stage mi ha poi permesso di capire la profonda differenza tra gli obiettivi che ci sono nel mondo del lavoro e
				quelli presenti nell'ambiente accademico. Infatti, per le aziende, ciò che conta davvero è il profitto, magari
				immediato. Questo è particolarmente vero per realtà di dimensioni contenute o che sono nate da poco. Questa
				differenza di visioni all'inizio mi ha un po' spaventato ma ben presto sono giunto a comprendere come mi fosse
				necessario giungere a compromessi in molte delle cose che facevo. Sono stato guidato gradualmente attraverso questo
				percorso dal mio tutor: in questo senso posso affermare che lo stage ha rappresentato un ponte tra università e
				mondo lavorativo.\\
				Infine, la possibilità stessa di entrare in azienda in modo guidato ha grande valore. Questo è particolarmente vero
				per studenti che — come il sottoscritto — erano completamente privi di esperienza lavorativa pregressa. Ritengo che
				questa possibilità abbia costituito, nel mio caso, l'elemento di maggiore crescita personale.
			\subsubsection{Prospettive ed idee differenti}
				L'ambiente e le circostanze nelle quali si è svolto lo stage sono stati fondamentali ai fini della buona riuscita
				dell'esperienza. L'ecosistema di aziende nel quale PastBook era inserita mi ha permesso di incontrare molte persone
				con conoscenze e capacità molto differenti: ho potuto condividere idee, chiedere aiuto o semplicemente ricevere un
				parere su alcune mie opinioni. Questa cosa mi ha permesso di ottenere numerosi spunti per lo sviluppo dei
				prodotti che stavo realizzando; inoltre, mi ha aiutato a cambiare prospettive e a vedere determinati aspetti da
				punti di vista differenti. Ho potuto apprezzare come idee strettamente informatiche si possano adattare ai bisogni
				di carattere economico, sociale e comunicativo.
	\section{Considerazioni personali sul corso di studi e il mondo del lavoro}
		Al termine della mia esperienza di stage, non posso affermare di aver rilevato importanti mancanze nella mia preparazione in
		rapporto a quanto richiesto dall'azienda. È evidente che vi siano marcate differenze tra il mondo del lavoro e quello accademico;
		ciò non è però da imputare ad una mancanza del percorso formativo universitario: esso deriva piuttosto dalla diversità di esigenze e
		di obiettivi dei due ambienti.\\
		Durante lo stage, ho lavorato con tecnologie a me sconosciute, che hanno richiesto un adeguato periodo di studio durante i primi
		giorni di attività. Ad ogni modo, un corso di laurea non dovrebbe avere la pretesa di formare gli studenti su ogni singola tecnologia
		— cosa tra l'altro impossibile — ma dovrebbe aiutarli ad affrontare le sfide derivanti dalla velocità del progresso: posso
		tranquillamente affermare che il corso di laurea in Informatica dell'Università di Padova riesce con successo nello scopo di
		fornire le basi teoriche che permettono di analizzare e affrontare numerose situazioni sconosciute.\\
		Naturalmente, l'inserimento nel mondo del lavoro può risultare difficile, qualsiasi sia la preparazione offerta agli studenti. In
		particolare, le aziende insistono molto su capacità pratiche da utilizzare in situazioni reali: l'università, in generale, fatica a
		rispondere a questi bisogni. Durante il corso di laurea in Informatica, tuttavia, ho potuto apprezzare gli insegnamenti ottenuti
		durante il corso di Ingegneria del Software: esso ha contribuito notevolmente nel minimizzare le difficoltà incontrate durante il
		processo di inserimento nel mondo del lavoro.\\
		Concludendo, sono convinto che affrontare con serietà il mio percorso di studi permetta di formare un profilo professionale di
		assoluto rilievo.
