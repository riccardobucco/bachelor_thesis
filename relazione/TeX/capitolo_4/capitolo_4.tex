\chapter{Resoconto retrospettivo}
	\section{Resoconto del raggiungimento degli obiettivi aziendali}
		\subsection{Funzionalità del prodotto}
			Elenco quali obiettivi posti inizialmente dall'azienda sono stati raggiunti e quali invece no. In questa sezione parlo delle
			funzionalità che l'azienda aveva deciso di inserire all'interno del prodotto che mi è stato assegnato.
		\subsection{Formazione e integrazione}
			Descrivo quali obiettivi riguardanti la mia formazione siano stati raggiunti. Tale formazione riguarda sia le tecnologie e
			gli strumenti che l'azienda mi ha messo a disposizione che il modo di lavorare che si è cercato di trasmettermi. Descrivo
			l'esito del tentativo dell'azienda di fare in modo che mi integrassi con il resto del team, portando nuove idee anche al di
			fuori dell'ambito in cui lavoravo. Descrivo il perchè posso affermare che certi obiettivi siano stati raggiunti
	\section{Bilancio dell'esperienza personale e professionale}
		\subsection{Obiettivi raggiunti}
			Al termine stage, posso affermare di aver raggiunto pienamente tutti gli obeittivi che mi ero posto prima di cominciare
			questa esperienza. Infatti, anche se ho dovuto affrontare alcune difficoltà, sono riuscito a sfruttare le varie situazioni
			a mio vantaggio, in modo tale da imparare il più possibile.\\
			\begin{center}
				\rowcolors{1}{azzurro}{azzurro_chiaro}
				\begin{tabular}[H]{p{0.25\textwidth} | p{0.60\textwidth}}
					Sviluppo capacità di collaborazione &
					Obiettivo raggiunto perché ho cooperato attivamente con colleghi in possesso di competenze completamente
					differenti dalle mie.\\
					\hline
					Potenziamento dell'autonomia &
					Obiettivo raggiunto grazie alla forte responsabilità assegnatami dall'azienda.\\
					\hline
					Accrescimento dello spirito critico &
					Obiettivo raggiunto in seguito alle numerose scelte che ho dovuto effettuare a livello tecnologico ed
					implementativo.\\
				\end{tabular}
			\end{center}
			Prima di cominciare lo stage, mi ero posto l'obiettivo di aumentare la mia scarsa capacità di collaborazione. A fine
			esperienza, posso dire che, da questo punto di vista, ho raggiunto risultati decisamente soddisfacenti. Infatti, ho
			avuto l'occasione di lavorare sia con colleghi della mia area lavorativa (sviluppatori e CTO) che con persone che avevano
			capacità completamente diverse dalle mie (designer e addetti al marketing). In qualsiasi caso, dovevo riuscire a raggiungere
			un compromesso tra la mia idea e quella del mio interlocutore, che spesso e volentieri era differente da quella che
			proponevo. In particolare, mi sono dovuo scontrare con esigenze completamente diverse dalle mie: ho dovuto imparare a dare
			loro importanza durante l'esecuzione delle mie attività.\\
			Un secondo obiettivo che mi ero posto era quello di potenziare la mia autonomia nel lavoro. Anche questo risultato è stato
			ampiamente raggiunto. Infatti, fin da subito mi è stata data la possibilità di effettuare scelte rilevanti durante lo
			sviluppo del prodotto. Inoltre, nonostante la mia scarsa esperienza lavorativa, ho svolto gran parte delle attività
			singolarmente, senza l'aiuto di nessuno. Ho infine potuto gestire il mio tempo come preferivo, a patto di rispettare certi
			vincoli o scadenze poste dall'azienda.\\
			L'ultimo mio scopo era quello di accrescere il mio spirito critico in qualsiasi cosa facessi. Posso dire di essere
			soddisfatto anche da questo punto do vista. Da una parte ho dovuto effettuare varie scelte implementative, valutando i pro e
			i contro di ogni possibile soluzione. Dall'altra ho dovuto affrontare numerosi problemi dovuti alle tecnologie utilizzate:
			ho fatto delle scelte che si sono rivelate essere pericolose, ma da questi errori ho imparato a porre più attenzione in quel
			che facevo.
		\subsection{Aspettative soddisfatte}
			Non tutte le aspettative che avevo prima di cominciare l'esperienza di stage si sono realizzate. Ciò deriva fondamentalmente
			dal fatto che non avevo esperienza lavorativa; non sapevo dunque come si sarebbe rivelato essere il mondo del lavoro.\\
			\begin{center}
				\rowcolors{1}{azzurro}{azzurro_chiaro}
				\begin{tabular}[H]{p{0.25\textwidth} | p{0.60\textwidth}}
					Rielaborazione, adattamento e approfondimento dei concetti studiati &
					Aspettativa soddisfatta solo in parte, a causa dell'approccio poco formale che ho dovuto utilizzare durante
					la mia permanenza in azienda.\\
					\hline
					Instaurazione di contatti per future collaborazioni &
					Aspettativa soddisfatta perché ho potuto incontrare e consocere molte persone con le quali scambiare idee e
					conoscenze.\\
				\end{tabular}
			\end{center}
			Innanzitutto, mi aspettavo che avrei fatto un grande uso delle nozioni e dei concetti appresi durante lo studio; credevo
			inoltre che un approccio analitico al problema sarebbe stato ben accetto da parte dell'azienda. Ciò è accaduto solo in parte.
			Infatti, posso affermare che l'azienda fosse completamente contraria all'approccio “scolastico” ai problemi: ciò
			che contava era il risultato, magari ottenuto in tempi molto brevi. Dunque, poco importava se a livello teorico un prodotto
			non fosse ben realizzato: l'imporante era che esso funzionasse e generasse profitti immediati. Questo modo di lavorare mi ha
			stupito parecchio e non mi è stato facile adattarmi. D'altra parte, però, posso affermare che la conoscenza acquisita
			durante gli studi mi è stata molto utile, come da aspettative. Infatti, nonostante i metodi imposti da PastBook, ho
			utilizzato le mie conoscenze in modo da prevenire molte delle situazioni pericolose in cui sarei potuto incorrere
			procedendo in modo poco ortodosso. In altre parole, ho utilizzato quanto sapevo per evitare errori piuttosto che per mettere
			in atto soluzioni ottimali, per le quali non avevo tempo.\\
			Una seconda aspettativa che avevo era di instaurare numerosi contatti per future collaborazioni. Da questo punto di vista
			sono stato contento. PastBook, infatti, è inserita in un contesto molto dinamico di startup e aziende consolidate, e questo
			mi ha dato l'opportunità di conoscere molte persone e aziende che cercassero personale al quale offrire concrete opportunità
			di lavoro.
		\subsection{Valore dello stage}
			\subsubsection{Ponte tra università e mondo del lavoro}
				Descrivo come lo stage abbia avuto un'importanza molto maggiore rispetto a quello che ero convinto avesse per
				immettermi nel mondo del lavoro. Descrivo come mi sia reso conto che, nonostante i numerosi strumenti che
				l'università mi offre, io abbia ancora molto da imparare (sia per quanto riguarda le conoscenze che per quanto
				riguarda la metodologia di lavoro). Indico infine come lo stage mi abbia permesso di capire che gli obiettivi nel
				mondo del lavoro sono molto differenti rispetto a quelli dell'ambito accademico: molto spesso sono dovuto giungere
				a compromessi e rinunciare a principi che ritenevo intoccabili in precedenza.
			\subsubsection{Prospettive ed idee differenti}
				Descrivo come l'ambiente (l'ecosistema di aziende nelle quali PastBook era inserita) e le circostanze nelle quali la
				mia esperienza si è svolta mi abbiano permesso di incontrare molte persone con le quali condividere idee, chiedere
				aiuto o semplicemente discutere su alcune nostre opinioni. Aggiungo come questa cosa mi abbia permesso di ottenere
				numerosi spunti sia per il prodotto che stavo realizzando che per aiutarmi a vedere molte cose sotto punti di vista
				differenti (non solo il punto di vista strettamente informatico, ma anche economico, sociale, comunicativo).
	\section{Considerazioni personali sul corso di studi e il mondo del lavoro}
		Espongo le mie considerazioni riguardanti quanto il corso di studi che ho frequentato sia appropriato in vista dell'inserimento
		nel mondo del lavoro. Descrivo i suoi pregi e i suoi difetti, includendo dove necessario dei consigli che mi sento di dare per
		poter migliorare l'attuale situazione.
