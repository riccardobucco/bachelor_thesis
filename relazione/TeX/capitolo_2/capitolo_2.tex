\chapter{Presentazione delle motivazioni, delle finalità e dei vincoli dello stage}
	\section{Gli stage nella strategia aziendale}
		\subsection{Motivazioni e obiettivi aziendali}
			Gli stage sono molto importanti all'interno della strategia aziendale. Essi hanno una triplice funzione:
			\begin{itemize}
				\item Innanzitutto costituiscono una preziosa risorsa per PastBook in quanto permettono la realizzazione di prodotti
				o servizi con un basso impatto sul budgeta a disposizione. Dal punto di vista dell'azienda, infatti, il costo di
				un'esperienza di stage si misura solo sulla base del tempo necessario per la sua formazione. Dunque, se uno stagista
				è in larga parte indipendente nello svolgere le proprie mansioni esso rappresenta un enorme valore aggiunto per
				PastBook.
				\item In secondo luogo, gli stage permettono all'azienda di confrontarsi con idee nuove e spesso originali. Queste
				possono essere tradotte in strategie o prodotti altamente innovativi da sfruttare sul mercato.
				\item Infine, grazie all'accurata selezione preliminare svolta da PastBook, le esperienze di stage rappresentano
				l'opportunità di entrare in contatto con giovani in possesso di spiccate capacità ed abilità che possano esssere
				inseriti in modo permanente in azienda.
			\end{itemize}
			Per quanto riguarda la mia personale esperienza di stage, l'obiettivo primario dell'azienda era quello di realizzare un
			prodotto altamente strategico per il mercato utilizzando un budget ridotto e senza dover distogliere la mente di altri
			sviluppatori da progetti già in fase avanzata. Allo stesso tempo, PastBook era fortemente interessata al contributo che avrei
			potuto portare in termini di idee e proposte riguardanti i vari aspetti dell'azienda con i quali sarei entrato in contatto.
		\subsection{Integrazione e formazione degli stagisti}
			Per poter raggiungere gli obiettivi prefissati l'azienda nomina un tutor (solitamente un lavoratore esperto che opera nello
			stesso contesto in cui è stato inserito lo stagista). Esso è chiamato a facilitare l'inserimento del giovane all'interno di
			PastBook e a seguire il suo percorso di crescita professionale.\\
			Per favorire una veloce integrazione nell'ambiente di lavoro, il tutor impiega parte del suo tempo a spiegare allo stagista
			come funzioni internamente l'azienda, quali siano gli obiettivi da raggiungere e che metodo di lavoro i dipendenti
			utilizzino. Lo stagista, inoltre, viene subito messo a contatto con gli altri dipendenti per permettere la conoscenza
			reciproca e lo scambio di idee.\\
			Il tutor aziendale si preoccupa anche di supportare il tirocinante durante l'acquisizione delle capacità e delle abilità
			necessarie all'attività lavorativa che gli è stata assegnata. In particolare, lo stagista è accompagnato lungo un percorso
			che ha il principale obiettivo di renderlo autonomo nello svolgimento delle sue mansioni. Per fare ciò, da una parte il tutor
			fornisce indicazioni e strumenti utili per assimilare velocemente le conoscenze necessarie, dall'altra esso interviene con
			azioni correttive piuttosto che preventive, in modo tale da permettere allo stagista di imparare dai suoi errori.\\
			Durante la mia esperienza l'azienda si è impegnata molto nel fare in modo che io mi ambientassi velocemente e imparassi ad
			adattarmi alla metodologia di lavoro interna. In particolare, sono stato invitato fin da subito	partecipare in modo attivo
			ai meeting durante i quali si propongono strategie da adottare nei vari ambiti in cui PastBook è impegnata. Inizialmente,
			inoltre, ho ricevuto aiuto concreto da parte del tutor nell'assimilare le nuove tecnologie a me richieste. Successivamente,
			invece, mi è stata concessa larga autonomia, intervallata da alcuni momenti di confronto che mi hanno permesso di individuare
			eventuali errori e di imparare da essi.
		\subsection{Gestione degli obiettivi di stage}
			PastBook è molto prudente nell'individuare gli obiettivi specifici che uno stagista deve raggiungere durante la sua
			esperienza. Questo approccio deriva dal fatto che spesso gli studenti non sono pronti per essere inseriti all'interno del
			mondo del lavoro e dunque non riescono a soddisfare pienamente aspettative troppo elevate.\\
			L'azienda, dunque, identifica inizialmente solo dei risultati minimi che il tirocinante deve conseguire: in assenza di essi
			il lavoro svolto durante lo stage non può essere ritenuto sufficiente. Con il passare del tempo, gli obiettivi sono adattati
			alle reali conoscenze e abilità dello stagista, oltre che alle nuove esigenze aziendali. Il tutor interviene personalmente
			solo nel caso in cui le difficoltà incontrate dal suo assistito nel raggiungere i risultati minimi siano eccessive.\\
			Per quanto riguarda la mia personale esperienza, ho cominciato con il concordare assieme al dirigente di PastBook gli
			obiettivi e i requisiti minimi che avrei dovuto auspicabilmente raggiungere. In seguito, dopo aver preso confidenza con il
			ruolo assegnatomi, ho proposto personalmente delle aggiunte a quanto pattuito: mi sono confrontato con il tutor per capire
			se esse avrebbero portato valore aggiunto all'azienda e se sarebbero state compatibili con risultati minimi che mi ero
			imposto di raggiungere. Talvolta il tutor ha notato l'irrealizzabilità o la complessità di alcuni obiettivi e ha deciso di
			ridimensionare le mie idee.
	\section{Descrizione del progetto di stage}
		\subsection{Prodotti}
			\subsubsection{PastBook Mobile SDK}
				Il primo prodotto che l'azienda intende realizzare tramite l'esperienza di stage è un SDK per
				Descrivo l'SDK che devo realizzare utilizzando le API e i services messi a disposizione da pastbook. Descrivo le
				funzionalità minime (e massime) che tale prodotto deve avere. Inquadro tale prodotto all'interno degli obiettivi
				aziendali futuri.
			\subsubsection{My Year Photo Book}
				Descrivo il prodotto principale che deve essere realizzato durante lo stage, un'applicazione per iphone e android
				che replichi le funzionalità offerte dall'azienda nel sito web. Descrivo come tale prodotto faccia uso dell'SDK
				di cui ho parlato in precedenza. Descrivo come tale prodotto sia fondamentale per la strategia aziendale e in
				generale per perseguire gli obietivi di cui ho parlato in precedenza.
		\subsection{Tecnologie}
			\subsubsection{Appcelerator Titanium}
				Appcelerator Titanium è un ambiente di sviluppo gratuito e open source che permette la realizzazione di applicazioni
				mobile native e cross-platform mediante l'uso del linguaggio Javascript.
				Esso è una combinazione dei seguenti componenti:
				\begin{center}
					\rowcolors{1}{azzurro_chiaro}{azzurro}
					\begin{tabular}[H]{p{0.25\textwidth} p{0.60\textwidth}}
						Titanium SDK 			& Insieme di strumenti basati su Node.js che si occupano di produrre
										  gli eseguibili per le varie piattaforme mobile a partire da codice
										  Javascript scritto dallo sviluppatore e in modo del tutto
										  trasparente. \emph{Titanium CLI} è l'interfaccia a linea di
										  comando utilizzata per accedere a questi strumenti.\\
						\hline
						Titanium API			& API basate su Javascript che permettono di accedere alle centinaia
										  di componenti	messe a disposizione dalle varie piattaforme mobile.
										  Queste funzionalità riguardano tanto la gestione dell'interfaccia
										  grafica quanto l'uso di sensori, dispositivi e moduli.\\
						\hline
						Appcelerator Studio		& IDE gratuito messo a disposizione dai creatori di Titanium che
										  rappresenta una valida alternativa grafica all'utilizzo di 
										  Titanium CLI. Esso permette agli sviluppatori di scrivere e 
										  testare le proprie applicazioni. Esso, inoltre, gestisce in modo
										  automatico gli aggiornamenti di Titanium SDK e dei vari moduli a
										  disposizione. Infine, tale ambiente di lavoro contiene numerosi
										  esempi e template pronti per l'uso che permettono a uno
										  sviluppatore principiante di imparare in fretta.\\
						\hline
						Moduli				& Estendono le funzionalità basilari offerte dalle Titanium API. Gli
										  sviluppatori possono liberamente creare nuovi moduli che si
										  adattino alle loro esigenze per poi aggiungerli all'ambiente di
										  sviluppo con l'aiuto di Appcelerator Studio.\\
						\hline
						Alloy Framework			& Framework che semplifica notevolmente il lavoro degli sviluppatori
										  in quanto permette loro di usare il pattern architetturale MVC: le
										  applicazioni possono essere realizzate separando separando
										  struttura, logica e stile. I linguaggi utilizzati per fare ciò
										  sono, rispettivamente, XML, Javascript e TSS (un linguaggio molto
										  simile a CSS per sintassi).\\
					\end{tabular}
				\end{center}
				Inoltre, Appcelerator mette a disposizione alcuni strumenti a pagamento che permettono di gestire e sviluppare
				applicazioni sfruttando il cloud. In particolare, i clienti abbonati hanno accesso alle funzionalità di
				\emph{Appcelerator Arrow} e \emph{Appcelerator Platform Services}.
				\begin{itemize}
					\item Il primo è un prodotto \emph{BaaS} che fornisce agli sviluppatori un modo per realizzare nuove API e
					collegare le loro applicazioni a servizi cloud quali gestione degli utenti, archiviazione di dati e
					notifiche push.
					\item Il secondo, invece, consiste in una dashboard che include:
					\begin{itemize}
						\item \emph{Appcelerator Test}, ovvero un abiente per l'esecuzione di test di integrazione e
						l'analisi dei risultati relativi;
						\item \emph{Appcelerator Performance}, un insieme di strumenti per monitorare le performance
						dell'applicazione e rilevare eventuali suoi crash;
						\item \emph{Appcelerator Analytics}, che permette di studiare il comportamento degli utenti (quante
						volte essi hanno installato l'applicazione, per quanto tempo mediamente la utilizzano, quali sono
						le azioni svolte più frequentemente etc.).
					\end{itemize}
				\end{itemize}
				La gamma di strumenti offerta da Appcelerator è molto ampia e permette agli sviluppatori di gestire l'intero ciclo di
				vita di produzione dell'applicazione. PastBook, tuttavia, ha scelto di non fare uso dell'ambiente cloud a pagamento:
				l'azienda, infatti, ha deciso di rivolgere la propria attenzione a servizi gratuiti che offrano le stesse
				funzionalità (in particolare Hockeyapp e Google Analytics).\\
				Infine, è utile sottolineare quali sono state le motivazioni che hanno spinto l'azienda a scegliere di sviluppare
				applicazioni mobile tramite l'uso di Titanium. In primo luogo, è stata scartata l'idea di utilizzare direttamente gli
				SDK di Android e iOS in quanto troppo differenti tra loro per poter essere imparati in tempo utile da uno stagista.
				In secondo luogo, è stato scartato \emph{PhoneGap} — il principale concorrente di Titanium — in quanto non permette
				di realizzare applicazioni realmente native: esso crea delle applicazioni web che hanno accesso a limitate
				funzionalità delle piattafore mobile e le incapsula all'interno di una finestra a tutto schermo di un browser.
			\subsubsection{Stripe}
				Stripe è un servizio di pagamento digitale tramite internet. I metodi di pagamento accettati sono molteplici: carte
				di credito, carte di debito, Apple Pay e Android Pay. Le funzionalità che esso mette a disposizione comprendono
				anche la gestione di sottoscrizioni ad abbonamenti, l'utilizzo di coupon e di periodi di prova per i prodotti.\\
				Esso funziona in modo molto simile a PayPal, suo principale concorrente. Gode tuttavia di alcuni vantaggi quali
				minori commissioni sui pagamenti ricevuti da altri utenti e una migliore documentazione delle API. Infine,
				contrariamente ad alcuni prodotti simili sul mercato, Stripe non richiede agli utenti di creare un account ma mette
				semplicemente a disposizione uno strumento grazie al quale ognuno può pagare con il proprio metodo preferito.
			\subsubsection{HockeyApp}
				HockeyApp è una piattaforma che mette a disposizione degli sviluppatori numerosi strumenti utili per le attività
				di collaudo. In particolare, essa offre un utile servizio di report dei crash: tramite l'uso di alcune semplici API è
				possibile fare in modo che l'applicazione invii informazioni utili circa il suo flusso di lavoro al quale è seguito
				un errore imprevisto. Tali dati sono utili basi di partenze per gli sviluppatori che intendono trovare e risolvere i
				bug presenti nel software. In secondo luogo, HockeyApp semplifica notevolmente la distribuzione dei prodotti:
				l'azienda può esercitare un accurato controllo sui beta tester che possono accedere alle varie versioni
				dell'applicazione e può gestire in modo intuitivo i feedback che essi riportano.\\
				HockeyApp si distingue da molti prodotti della concorrenza per il fatto di essere completamente cross-platform:
				questo permette un suo uso su tutti i maggiori sistemi operativi mobile. Esiste inoltre un modulo per Titanium che
				mette a disposizione tutte le maggiori funzionalità di HockeyApp sulla piattaforma di Appcelerator.
			\subsubsection{Google Analytics e Google Tag Manager}
				Google Analytics è un servizio di analisi dei dati gratuito di Google che consente di trattare delle dettagliate
				statistiche sui fruitori di un prodotto. Esso si rivolge principalmente agli esperti di marketing che vogliono
				studiare il comportamento degli utenti: cosa essi fanno, quanto tempo è loro necessario per compiere una certa
				azione, quali parti non vengono utilizzate etc. Gli innumerevoli strumenti messi a disposizione da Google Analytics
				funzionano grazie a dei frammenti di codice inseriti nel prodotto da analizzare che consentono di raccogliere dati
				relativi all'esperienza degli utenti.\\
				Google Tag Manager, invece, è uno strumento gratuito che consente di configurare le applicazioni per dispositivi
				mobile: esso permette quindi di liberare gli sviluppatori dal compito di riscrivere il codice e di avviare un nuovo
				tedioso ciclo di rilascio della nuova versione. Tale strumento, inoltre, offre la possibilità di registrare in modo
				semplice alcuni eventi che avvengono all'interno di un'applicazione: i dati raccolti sono inviati in modo
				automatico ai servizi di Google Analytics.
		\subsection{Scadenze}
			L'applicazione My Year Photo Book riveste un ruolo fondamentale all'interno delle strategie commerciali di PastBook. Il suo
			rilascio, infatti, è previsto per le vacanze natalizie: questo è il più importante periodo dell'anno per l'azienda, che si
			aspetta di ampliare la vendita dei propri prodotti anche agli utenti mobile, incrementando così le entrate. Prima di essere
			rilasciata, inoltre, l'applicazione deve essere sottoposta a numerosi test perchè l'azienda desidera fornire ai clienti un
			prodotto che li soddisfi pienamente. PastBook, infine, sottopone i propri prodotti alla costante approvazione di alcuni
			investitori e mentori.\\
			Date queste premesse, l'azienda mi ha imposto di concludere alcune delle attività collegate alla mia esperienza di stage
			entro termini prestabiliti e molto rigidi:
			\begin{itemize}
				\item Ogni settimana devo consegnare un prototipo che contenga tutte le funzionalità sviluppate. Questi prototipi
				servono per aggiornare in modo costante investitori, mentori e collaboratori.
				\item Una versione dell'applicazione che comprenda tutte le funzionalità richieste deve essere pronta entro
				la sesta settimana di stage. Tale applicazione sarà consegnata ai beta tester affinchè la collaudino e offrano
				all'azienda preziosi feedback.
				\item Una versione stabile e senza errori evidenti deve essere pronta entro l'ottava settimana di stage. Essa deve
				essere inviata a giornalisti e blogger in modo tale che ne venga pubblicizzato il rilascio definitivo.
			\end{itemize}
		\subsubsection{Metodologia di lavoro}
			Descrivo i vincoli metodologici, ovvero la metodologia di lavoro che sono tenuto a seguire in azienda per poter collaborare
			al meglio con gli altri membri del team. Descrivo come ci si aspetta che io lavori per poter controllare che gli obiettivi
			vengano raggiunti e le scadenze (di cui ho parlato in precedenza) siano rispettate.
	\section{Scelta dello stage}
		\subsection{Obiettivi personali}
			\subsubsection{Sviluppo della capacità di collaborazione}
				Saper collaborare — ovvero essere capaci di lavorare in gruppo — è una competenza indispensabile per un giovane alle
				prime esperienze lavorative. Durante il corso dei miei studi, tuttavia, ho avuto poche occasioni durante le quali
				poter migliorare in questa abilità.\\
				Dunque, il principale obiettivo che mi sono imposto di raggiungere durante la mia esperienza di stage era quello di
				poter imparare cosa significasse cooperare con i colleghi per raggiungere uno scopo comune. In particolare, ero molto
				interessato ad aumentare la mia capacità di confronto con persone in possesso di competenze, necessità e conoscenze
				diametralmente opposte alle mie.
			\subsubsection{Potenziamento dell'autonomia}
				Ritengo che l'autonomia sia una competenza molto importante nel mondo del lavoro almeno per due ragioni:
				\begin{itemize}
					\item A una persona autonoma possono essere delegate responsabilità: essa ottiene poteri oltre che doveri.
					Questo permette di liberare spazi e tempo a coloro che si occupano del controllo dei dipendenti.
					\item Una persona autonoma gode spesso di libertà nelle scelte e può essere più creativa nelle attività a
					essa assegnate. Questo permette di aumentare la soddisfazione del lavoratore e di migliorare la sua
					produttività. 
				\end{itemize}
				Durante il corso dei miei studi ho sviluppato gradualmente questa capacità. Ritenevo tuttavia di doverla potenziare
				per poter essere davvero competitivo nel mondo del lavoro: lo stage rappresentava un'utile possibilità per realizzare
				questo obiettivo.
			\subsubsection{Accrescimento dello spirito critico}
				Il terzo e ultimo obiettivo personale che mi sono imposto di raggiungere durante la mia esperienza di stage era
				quello di sviluppare la mia propensione a esaminare ogni situazione o concetto in profondità. Ritengo che nel mondo
				dell'informatica la capacità di pensare in modo critico sia assolutamente fondamentale. Infatti, le tecnologie a
				nostra disposizione aumentano continuamente a dismisura e c'è dunque bisogno di grande accortezza nell'applicare
				criteri per valutare quanto ciascuno strumento sia efficiente ed efficace per lo scopo che si intende raggiungere.\\
				Lo stage rappresentava per me l'opportunità perfetta di applicare gli strumenti ottenuti durante gli studi per
				valutare in modo intelligente i concetti sconosciuti con cui sarei entrato in contatto.
		\subsection{Motivazioni per cui ho intrapreso lo stage}
			\subsubsection{Ambiente internazionale}
				Ho scelto di effettuare la mia esperienza di stage presso PastBook principalmente a causa dell'ambiente fortemente
				internazionale nel quale l'azienda è inserita. Essa, infatti, comprende persone con competenze, obiettivi, passioni
				e culture completamente differenti dalle mie. Più in generale, sono stato attratto dalla possibilità di intraprendere
				un'esperienza che andasse oltre un mio graduale inserimento nel mondo del lavoro: volevo entrare in contatto con
				l'immenso ecosistema di startup presente ad Amsterdam e con le relative idee ed opinioni.\\
				Questa motivazione — rivelatasi decisiva nella mia scelta finale — è nata da uno degli obiettivi che mi ero imposto
				di raggiungere durante lo stage: imparare a collaborare. Infatti, ero convinto che la presenza di persone con
				background molto differenti mi avrebbe stimolato molto a migliorare il mio modo di lavorare con i colleghi.
			\subsubsection{Utilizzo di Titanium}
				Un secondo motivo che mi ha spinto a scegliere di svolgere lo stage presso PastBook riguarda l'aspetto tecnologico.
				Infatti, sono stato subito attratto dal framework Titanium che mi era stato proposto di utilizzare. In particolare,
				mi piaceva l'idea di poter utilizzare Javascript — un linguaggio che avevo già usato in molti altri ambiti — per
				poter realizzare applicazioni native cross-platform.\\
				Prima di effettuare la mia scelta mi sono informato molto sulle caratteristiche di questa tecnologia. Nonostante
				abbia rinvenuto alcune critiche da parte degli utenti, sono stato spinto dal desiderio di utilizzare uno strumento
				che fosse ancora in fase di elaborazione: volevo infatti mettere alla prova il mio spirito critico nel giudicare e
				valutare ciò che avrei utilizzato.
			\subsubsection{Dimensione ridotta dell'azienda e informalità dei rapporti}
				L'ultimo fattore che ho tenuto in considerazione durante la mi scelta ha riguardato le dimensioni di PastBook. Ero
				infatti abbastanza intimorito dal pensiero di dover essere inserito all'interno di un'azienda medio/grande: la paura
				era quella di non poter essere sufficientemente seguito durante il mio percorso e nei momenti di difficoltà. Inoltre,
				temevo di non poter avere una visione d'insieme dell'ambiente in cui ero inserito, isolando la mia conoscenza al
				solo reparto dedicato allo sviluppo. Questo mi avrebbe precluso la possibilità di imparare come competenze e ruoli
				differenti coesistano e collaborino all'interno di un'azienda.\\
				Il fatto che le dimensioni di PastBook siano notevolmente ridotte mi avrebbe inoltre permesso di ottenere grande
				autonomia nelle mie mansioni. Infatti, gli sviluppatori in azienda erano pochi e con compiti in larga parte
				completamente differenti dai miei: lo svolgimento del lavoro spettava unicamente a me e questo mi permetteva di
				raggiungere il mio obiettivo di potenziare la capacità di prendere decisioni e lavorare autonomamente.\\
				Infine, l'azienda ha una politica interna che è fortemente basata su rapporti informali fra colleghi: questo mi
				avrebbe permesso di sentirmi più libero nel chiedere consigli nei momenti di difficoltà. Ho preferito questa opzione
				come prima esperienza lavorativa, evitando ambienti troppo formali e soggetti a regole eccessivamente rigide.
		\subsection{Aspettative}
			\subsubsection{Rielaborazione, adattamento e approfondimento dei concetti studiati}
				Prima di cominciare la mia esperienza di stage ero convinto che avrei fatto grande uso dei concetti e delle nozioni
				imparate durante il corso dei miei studi universitari. In particolare, mi aspettavo che l'azienda avrebbe accolto
				in modo favorevole un approccio fortemente analitico ai problemi che avrei dovuto affrontare di volta in volta. Il
				modo di lavorare che intendevo seguire comprendeva:
				\begin{itemize}
					\item studio del problema da un punto di vista formale;
					\item individuazione di concetti studiati che sarebbero potuti essere utili e seguente adattamento al
					problema;
					\item approfondimento dei concetti studiati, sia dal punto di vista teorico che pratico.
				\end{itemize}
				In tutto scenario, il tutor assegnatomi mi avrebbe aiutato e indirizzato con utili consigli ed eventuali correzioni.
			\subsubsection{Instaurazione di contatti per future collaborazioni}
				Una delle principali aspettative che avevo prima di cominciare riguardava l'ambiente fortemente innovativo nel quale
				PastBook è inserita. Infatti, l'azienda è nata all'interno di un contesto molto dinamico formato da startup e imprese
				ormai consolidate, con il supportato di mentori e il finanziamento di ricchi investitori: i miei piani prevedevano
				l'instaurazione di numerosi contatti sui quali basare future collaborazioni. Oltre a ciò, tale esperienza sarebbe
				stata l'occasione perfetta per poter entrare in contatto con aziende che cercassero personale giovane e motivato al
				quale offrire concrete opportunità di lavoro.
