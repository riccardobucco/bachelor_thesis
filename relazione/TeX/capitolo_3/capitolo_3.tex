\chapter{Sintesi delle attività svolte durante lo stage}
	\section{Metodologia agile}
		\subsection{Pianificazione di iterazioni in modo adattivo}
			Durante lo svolgimento dello stage mi sono fortemente adattato al modo di lavorare tipicamente utilizzato in azienda.
			PastBook, infatti, utilizza una metodologia \emph{agile} per organizzare l'intero team e, in particolare, il gruppo di coloro
			che sono dedicati alla produzione del software. Questo approccio è particolarmente adatto all'azienda a causa delle sue
			ridotte dimensioni.\\
			La prima cosa che ho notato nel modo di lavorare dei miei colleghi è stato il fatto che essi inizino a svolgere le loro
			attività senza avere ben chiari tutti i requisiti che il prodotto dovrebbe soddisfare entro la fine del progetto.
			Essi, in particolare, fanno una pianificazione molto generica all'inizio di ciascun mese, concentrandosi più sugli
			obiettivi che dovrebbero raggiungere e sui requisiti da soddisfare piuttosto che sulle attività da eseguire per poter
			adempiere ai loro compiti. Di settimana in settimana, poi, ogni gruppo di progetto pianifica nel dettaglio tutto il lavoro
			che si dovrebbe svolgere. I progetti aziendali, dunque, procedono in modo adattivo piuttosto predittivo: il team preferisce
			adattare il modo di lavorare ai risultati ottenuti piuttosto che prevedere a priori ogni singola eventualità nel dettaglio.\\
			Per quanto mi riguarda, ho innanzitutto accordato con i mio tutor quali sarebbero stati gli obiettivi e i requisiti minimi
			da raggiungere durante i due mesi di stage. In seguito ho pianificato con cadenza settimanale i singoli task che avrei
			dovuto svolgere: ho cercato di rendere più granulare possibile ogi singolo comèito, in modo tale che non richiedesse più di
			mezz'ora per essere portato a termine. Una volta terminata l'iterazione ho analizzato i risultati raggiunti per capire se
			fossero sopraggiunte nuove difficoltà e quale sarebbe stato il modo migliore di aggiornare i requisiti.
		\subsection{Coinvolgimento diretto e continuo del cliente}
			Io e il team di sviluppo di cui facevo parte abbiamo coinvolto in modo continuo i clienti — che nel nostro caso
			corrispondevano al CTO e al dirigente — durante l'esecuzione di tutte le nostre attività. Essi hanno fornito feedback utili
			per capire pregi e difetti del prodotto: questo mi ha permesso di adattare il mio lavoro alle richieste dei clienti oltre che
			alle difficoltà incontrate.\\
			Con cadenza settimanale ho partecipato a brevi riunioni con il mio gruppo di lavoro e i clienti stessi. Durante questi
			meeting abbiamo discusso apertamente sullo stato di avanzamento del prodotto. Abbiamo inoltre evidenziato problemi che
			hanno rallentato lo sviluppo e che si sono rivelati essere più ardui del previsto: in questi casi abbiamo concordato delle
			modifiche ai requisiti richiesti. Eventuali nuove idee sono state vagliate in base al loro costo.
		\subsection{Consegne frequenti}
			Una mia costante preoccupazione durante l'intera durata dello stage è stata quella di effettuare numerosi rilasci di versioni
			intermedie del software prodotto. Questa prototipazione frequente è stata richiesta esplicitamente dal dirigente aziendale:
			egli intendeva testare costantemente le nuove funzionalità aggiunte di volta in volta.\\
			In particolare, quando ho cominciato lo sviluppo di My Year Photo Book ero tenuto a consegnare una versione funzionante
			dell'applicazione al termine di ogni giornata, salvo rare eccezioni. Effettuare rilasci frequenti di prototipi mi ha permesso
			di godere di alcuni vantaggi:
			\begin{itemize}
				\item Nel caso di eccessive difficoltà riscontrate durante lo sviluppo avevo sempre a disposizione una o più basi
				solide dalle quali sarei potuto ripartire. L'aggiunta di nuove caratteristiche non stravolgeva l'intero funzionamento
				dell'applicazione.
				\item Ero costretto a prestare la massima attenzione a non introdurre errori. La loro eventuale correzione aveva
				priorità assoluta poichè senza essa non avrei potuto ottenere un prototipo funzionante. Di conseguenza, l'attività di
				mantenimento dell'applicazione non è stata particolarmente impegnativa.
				\item Ho ricevuto numerosi feedback dai clienti che hanno avuto la possibilità di provare l'applicazione. Questi
				hanno in larga parte aiutato — e in parte sostituito — le mie attività di verifica e collaudo.
			 \end{itemize}
			 Lo svantaggio introdotto da questo approccio ha riguardato perlopiù la lentezza alla quale ero costretto a procedere.
		\subsection{Pair programming}
			Una delle tecniche più utilizzate in azienda è il \emph{pair programming}. Essa prevede che due programmatori lavorino
			ad una stessa postazione di lavoro: uno dei due scrive il codice ed ha l'obiettivo principale di realizzare una soluzione
			funzionante del problema in considerazione, l'altro svolge un ruolo di revisione simultanea del codice e gli è lasciato il
			compito di segnalare errori o proporre strategie alternative di soluzione.\\
			Ho partecipato a molte sessioni di pair programming, sia in qualità di programmatore che di osservatore, principalmente con
			il mio tutor. Ho sfruttato questa tecnica soprattutto nei momenti di difficoltà o comunque quando non era ben chiaro al
			gruppo di lavoro quale fosse la miglior strategia da perseguire. Il pair programming, in particolare, si è rivelato utile o
			addirittura indispensabile in due tipi di situazione:
			\begin{itemize}
				\item Tale tecnica mi ha aiutato durante le fasi iniziali dello sviluppo dei nuovi componenti. In azienda, infatti,
				non è previsto un vero e proprio momento in cui si progetta l'architettura del software e questi momenti di
				confronto con un programmatore più esperto mi hanno aiutato a individuare quali fossero le soluzioni più opportune
				da applicare ai vari casi.
				\item Questa tecnica è stata inoltre indispensabile durante la risoluzione di problemi particolarmente ardui alla
				quale partecipava anche il mio tutor. Infatti, non essendo previsa l'attività di stesura della documentazione del
				software, era difficile per una persona che non avesse mai programmato direttamente comprendere a fondo tutto il
				codice scritto da me. In questi casi ho dunque agito da supervisore, controllando che eventuali modifiche apportate
				non stravolgessero il funzionamento dell'applicazione e spiegando le ragioni per le quali avevo agito in un
				determinato modo piuttosto che in un altro.
			\end{itemize}
	\section{Attività preparatorie e di accompagnamento}
		\subsection{Configurazione dell'ambiente di lavoro}
			\subsubsection{Configurazione delle macchine virtuali}
				I primi giorni della mia esperienza di stage sono stati dedicati all'installazione e alla configurazione
				dell'ambiente con il quale avrei lavorato per i rimanenti due mesi. Quest'attività è stata costantemente
				supervisionata dal CTO di PastBook, che si è accertato che tutti gli strumenti necessari funzionassero in modo
				corretto.\\
				Ho cominciato con l'installazione della sandbox tramite la quale avrei potuto eseguire il collaudo dei miei prodotti
				senza dover interfacciarmi direttamente con i server aziendali. La configurazione di questo strumento è stata
				abbastanza agevole in quanto in azienda era disponibile una guida realizzata dal CTO che illustrava tutti i passi
				necessari. Ho contribuito al miglioramento della guida nei passaggi non sufficientemente chiari.
			\subsubsection{Installazione di Titanium e delle relative dipendenze}
				Descrivo come ho proceduto all'installazione di tutti gli strumenti messi a disposizione da Appcelerator. Racconto
				perchè alcuni strumenti siano stati scartati in partenza mentre altri siano stati ritenuti utili. Racconto come a
				circa metà stage abbia dovuto aggiornare il framework a causa di esigenze aziendali, dovendo ritoccare parte della
				configurazione iniziale (oltre che pate del codice)
			\subsubsection{Scelta e configurazione degli emulatori di dispositivi mobile}
				Descrivo i criteri sui quali mi sono basato per scegliere gli emulatori che avrei dovuto usare per testare
				l'applicazione. Descrivo i pro e i contro di ognuno di essi, basati su test effettuati da me stesso. Descrivo come
				li abbia configurati in modo tale che funzionino con le macchine virtuali precedentemente installate.
		\subsection{Studio e preparazione personale}
			Descrivo come inizialmente abbia imparato ad utilizzare gli strumenti messi a disposizione dall'azienda e come abbia studiato
			il framework titanium e tutti i suoi vari elementi. Descrivo come la parte principale di tutto ciò sia stata dedicata alla
			comprensione del modulo Alloy (MVC con Titanium). Racconto come la formazione comprendesse anche l'uso di tali
			strumenti per la produzione di piccoli esempi.
	\section{Sviluppo di PastBook Mobile SDK}
		\subsection{Autenticazione a Facebook e Instagram}
			\subsubsection{Studio del dominio e analisi dei requisiti}
				Descrivo come abbia cominciato preoccupandomi innanzitutto di ottenere l'autenticazione a facebook e instagram prima
				di svolgere qualsiasi altro compito. Descrivo quali sono stati i requisiti che ho individuato (riguardanti tutti i
				possibili scenari dell'autenticazione e i modi in cui gestirli). Racconto poi del seguente studio dell API messe a
				disposizione daai due social network.
			\subsubsection{Implementazione dell'autenticazione a Facebook tramite un modulo Titanium}
				Racconto di come ho deciso di utilizzare il modulo per Facebook messo a disposizione da Appcelerator. Descrivo i
				problemi riscontrati nel suo utilizzo (presenza di bug che mi impedivano lo sviluppo cross-platform e il
				raggiungimento di tutti i requisiti individuati in precedenza). Descrivo i modi in cui ho testato tale autenticazione
				e i compromessi ai quali sono giunto in accordo con l'azienda.
			\subsubsection{Implementazione dell'autenticazione a Instagram tramite API native}
				Racconto di come non esistano moduli degni per l'autenticazione tramite Instagram e di come abbia dunque deciso di
				implementare tale cosa facendo uso direttamente delle API messe a disposizione dal social network. Descrivo come la
				mancanza di un'autenticazione SSO mi abbia indotto a scegliere un approccio diverso rispetto a quanto fatto in
				precedenza (uso di una pagina webview invece che di una schermata nativa).
		\subsection{Interfaccia per API e services aziendali}
			\subsubsection{Studio del dominio e analisi dei requisiti}
				Descrivo come in prima istanza mi sia dedicato allo studio delle API (ben documentate) e dei services (documentazione
				non presente) messi a disposizione dall'azienda. Descrivo come abbia poi cominciato l'analisi dei requisiti,
				concentrandomi in particolare su quali di questi servizi sarebbero potuti servire durante lo sviluppo di applicazioni
				(e se eventualmente fosse il caso di aggregare le loro funzionalità). Descrivo i risultati ai quali sono pervenuto.
			\subsubsection{Progettazione, implementazione e relativi problemi}
				Descrivo come inizialmente la progettazione sia stata quasi nulla a causa della fretta da parte dell'azienda di avere
				un prototipo funzionante e di come abbia cominciato subito con l'implementazione. Descrivo come, facendo ciò, si sia
				arrivati ad avere un modulo funzionante ma con un sacco di codice ripetuto e difficilmente verificabile. Descrivo
				come abbia in seguito speso varie energie per ripensare all'architettura dell'SDK. Descrivo i risultati ai quali sono
				pervenuto e porto un esempio di come abbia eliminato molto del codice doppio o superfluo.
			\subsubsection{Gestione degli errori}
				Descrivo in che modo vengono gestiti tutti i possibili errori che possono avvenire durante l'uso dei servizi e delle
				API messe a disposizione dall'azienda. Racconto di come alcuni di essi siano gestiti direttamente dall'SDK e di come
				la gestione di alcuni sia invece assegnata a chi utilizza le API. Descrivo infine i test ai quali ho sottoposto
				l'SDK per controllare che le chiamate fossero corrette.
		\subsection{Launcher configurabile}
			\subsubsection{Analisi del launcher preesistente}
				Descrivo cos'è un launcher per l'azienda (serie di chiamate ai services aziendali con il fine comune di creare un
				photo book, con ogni chiamata che restituisce un risultato a partire dai dati provenienti dalle chiamate precedenti).
				Descrivo i limiti di quello già posseduto e implementato dall'azienda (principalmente esso non funziona se vengono
				cambiati i servizi chimati, ovvero non è configurabile ed è troppo specifico).
			\subsubsection{Progettazione e implementazione di un nuovo launcher}
				Descrivo l'algoritmo di maggior rilevanza sviluppato durante lo stage. Esso rappresenta un launcher per qualsiasi
				futura applicazione sviluppata dall'azienda. Il grande pregio è di essere completamente configurabile: entro certi
				vincoli, l'azienda può decidere in qualsiasi momento di cambiare il modo in cui un photo book viene creato senza
				dover per forza modificare l'applicazione. Descrivo cosa esso può fare ed entro quali vincoli opera. Mi concentro
				in particolare sulla nuova funzionalità aggiunta rispetto al suo predecessore: il fatto di poter essere messo in
				pausa in qualsiasi momento
			\subsubsection{Verifica e correzione degli errori}
				Descrivo i principi secondo i quali sono stati preparati i test. Descrivo la tipologia di test effettuati e i loro
				limiti. Descrivo come si sia rivelata molto più utile l'anaisi statica, in quanto i malfunzionamenti erano tutti
				casi troppo particolari per essere testati in modo dinamico (principalmente dovuti a usi errati della concorrenza).
	\section{Sviluppo di My Year Photo Book}
		\subsection{Analisi dei requisiti}
			\subsubsection{Produzione di alcuni wireframe}
				Descrivo le modalità con le quali è stata condotta l'analisi dei requisiti dell'applicazione che è stata fatta
				inizialmente. Descrivo i risultati raggiunti e gli strumenti che sono stati utilizzati per produrre la
				documentazione, che consiste perlopiù in wireframe e diagrammi di attività.
			\subsubsection{Design e marketing}
				Descrivo come una parte fondamentale dell'analisi dei requisiti (svolta verso la fine dello stage) abbia riguardato
				il design che l'applicazione avrebbe dovuto avere. Descrivo come ho collaborato con un esperto per poter capire
				cosa fosse possibile e sensato aggiungere. Descrivo inoltre come durante l'analisi molti dei particolari aggiunti
				riguardassero puro marketing, e come abbia dovuto scendere a compromessi con principi di usabilità e accessibilità.
				Descrivo come si desiderasse che l'utente non si accorgesse dei tempi di attesa (quindi lo si voleva distrarre) e
				allo stesso tempo arrivasse con un singolo click alla creazione e visualizzazione del photo book, in linea con
				gli obiettivi aziendali. Descrivo tutte le opzioni che sono state considerate a tal proposito.
		\subsection{Progettazione}
			\subsubsection{Modularizzazione della struttura dell'applicazione}
				Descrivo come si è svolta l'attività di progettazione e in particolare riporto alcuni dei risultati. Descrivo come
				l'applicazione sia stata suddivisa in moduli riutilizzabili in più schermate e anche in applicazioni simili. 
				Descrivo come siano stati sfruttati alcuni pattern per semplificare la struttura, la comprensibilità e l'utilizzo
				delle varie classi.
			\subsubsection{Utilizzo delle funzioni di PastBook Mobile SDK}
				Descrivo come l'applicazione sia pensata per far pesante uso delle funzioni messe a disposizione dall'SDK creato in
				precedenza. Descrivo come l'SDK eviti molto lavoro a chi deve sviluppare una nuova applicazione per l'azienda.
		\subsection{Implementazione delle schermate e dei relativi controller}
			\subsubsection{L'uso del framework MVC Alloy}
				Descrivo come ho utilizzato il framework ALloy (messo a disposizione da Appcelerator) per implementare in modo
				facile il pattern MVC. Descrivo i vantaggi derivanti dall'utilizzo di tale framework rispetto all'uso della versione
				base di Titanium. Descrivo le motivazioni della mia scelta.
			\subsubsection{Creazione di un Photo Book}
				Descrivo la schermata iniziale dell'applicazione. Descrivo come essa è stata implementata e quali sono i principali
				elementi presenti in essa. Descrivo come molti di essi siano riutilizzabili altrove (in altre schermate o anche
				altre applicazioni). Descrivo le principali animazioni che sono state implementate in seguito all'analisi dei
				requisiti dedicata.
			\subsubsection{Intrattenimento dell'utente}
				Descrivo la schermata dedicata all'intrattenimento dell'utente durante la creazione di un photo book. Descrivo come
				le finalità di tale schermata avessero scopi di puro marketing (si vuole distrarre l'utente durante l'attesa).
				Descrivo come l'impossibilità di implementare alcuni particolari (a causa di bug del framework) abbiano comportato
				dei cambiamenti nell'analisi dei requisiti. Descrivo il risultato al quale si è arrivati alla fine. Descrivo i
				problemi che esso ha causato sia in termini di prestazioni su dispositivi lenti che in termini di memoria utilizzata.
				Descrivo come tale schermata sia stata ottimizzata.
			\subsubsection{Visualizzazione del Photo Book}
				Descrivo la schermata dedicata alla visualizzazione del photo book. Descrivo come sia stato implementato il libro
				3d sfogliabile dall'utente. Descrivo i problemi che l'implementazione di tale schermata ha comportato (dovuti
				soprattutto all'orientamento del telefono). Descrivo come tali problemi sono stati affrontati e risolti. Descrivo
				infine la possibilità per l'utente di condividere il photo book creato e di votare l'applicazione sullo store
				(solo in determinate situazioni, ovvero quando si presume che all'utente l'applicazione sia piaciuta).
			\subsubsection{Modifica del Photo Book}
				Descrivo le schermate grazie alle quali un utente può modificare i propri photo book. Descrivo come la sua
				implementazione sia stata semplice grazie all'uso dell'SDK. Descrivo come gestisce e corregge eventuali errori senza
				che l'utente se ne accorga. Descrivo come sia stato possibile riutilizzare molti degli elementi presenti nella
				schermata di intrattenimento.
			\subsubsection{Checkout}
				Descrivo la schermata nella quale l'utente può comprare il photo book creato. Descrivo perchè ho deciso assieme
				all'azienda di implementarla facendo largo uso di una webview invece che di elementi nativi.
		\subsection{Verifica dei risultati ottenuti}
			Descrivo come sia stata effettuata l'attività di verifica sull'applicazione. Spiego che la parte principale è stata
			svolta mediante l'utilizzo di emulatori e grazie a una serie di test decisi a priori e documentati (seppur non
			automatizzati). Spiego inoltre come consegnassi quasi giornalmente un prototipo a 5-6 persone che si preoccupavano
			di provare l'applicazione e di segnalare malfunzionamenti o rallentamenti. Spiego come ho implementato un sistema
			che permetteva ai tester di essere più precisi possibile riguardo gli errori che erano riportati.
	\section{Mantenimento di My Year Photo Book}
		\subsection{Rilascio dell'applicazione}
			Descrivo in che modo l'applicazione è stata resa disponibile al pubblico per il beta testing. Descrivo gli strumenti
			utilizzati e gli utenti ai quali è stato dato l'accesso alla preview.
		\subsection{Raccolta dei dati}
			Descrivo in che modo e con quali tecnologie si sono raccolti i dati di utilizzo da parte degli utenti e i dati riguardanti
			eventuali crash dell'applicazione. Descrivo in base a quali principi si è scelto di raccogliere certi dati piuttsto che altri
		\subsection{Analisi dei dati}
			Descrivo in che modo ho proceduto all'analisi dei dati provenienti dagli utenti che stavano utilizzando l'applicazione in
			anteprima. Descrivo come grazie ai dati e ai feedback ho potuto trovare e correggere alcuni errori. Descrivo a quali
			conclusioni sono arrivato studiando i dati di utilizzo a disposizione e come ho proposto all'azienda quali sarebbero stati
			gli aspetti migliorabili da un punto di vista della soddisfazione degli utenti
	\section{Analisi dei principali problemi}
		\subsection{Criticità riscontrate usando Titanium}
			\subsubsection{Documentazione scadente e community limitata}
				Racconto come un forte limite di tale prodotto sia la presenza di una documentazione scritta male, molto variabile
				e talvolta anche errata. A questo si aggiunge il fatto che la community online sia di dimensioni molto ridotte.
				Tutto ciò comporta che spesso lo sviluppatore si trovi a dover risolvere i problemi che sorgono senza poter contare
				sull'aiuto di questi due importanti strumenti.
			\subsubsection{Sviluppo cross-platform impossibile}
				Descrivo come quello che dovrebbe essere un punto di forza di tale tecnologia (poter sviluppare in modo
				cross-platform) non sia in realtà del tutto verificato. Porto la mia esperienza dicendo come dopo le difficoltà
				iniziali l'azienda abbia deciso di dividere in due parti lo sviluppo dell'applicazione Android e di quella iOS.
			\subsubsection{Presenza di bug}
				Descrivo come durante la mia esperienza mi sia capitato di trovare bug abbastanza seri e che rendevano quasi del
				tutto inutilizzabili alcune API messe a disposizione dall'SDK.
		\subsection{Difficoltà dovute al metodo di lavoro utilizzato}
			\subsubsection{Rallentamenti causati dalla necessità di numerosi prototipi}
				Descrivo come il fatto che mi fossero richiesti prototipi funzionanti rallentasse talvolta il mio lavoro. Se da una
				parte infatti avevo la possibilità di far testare il prodotto dall'altra dovevo occupare parte del tempo a renderlo
				funzionante fin da subito per poter essere testato. Porto degli esempi.
			\subsubsection{Modifiche all'architettura software}
				Racconto come abbia speso abbastanza energie nel cambiare l'architettura stessa del software che stavo producendo
				a causa del fatto che non è stato dedicato abbastanza tempo alla progettazione architetturale. Descrivo quali sono
				stati gli svantaggi sia dal mio punto di vista che da quello aziendale.
